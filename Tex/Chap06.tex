\chapter{图像的去噪方法}
\label{chap:Segmentation}


\section{初始位置定位和特征点标注}

\section{结合卷积网络特征的形状对齐模型} 
\subsection{超声组织特征纹理特异性灰度归一化}
\subsection{结合不同外观特征的全局AAM}
\section{实验结果分析和讨论}
\section{小结与讨论}

本文利用深度学习来解决医学图像计算机辅助检测问题,设计并验证了自动检测MRI短轴和超声心动图中LV长轴切面的方法,在通用物体检测Faster RCNN框架的基础上,针对RPN引入空间变换,结合带朝向损失的多任务损失,探索解决图像平面内物体旋转角度检测的问题,并利用困难样例挖掘策略加快迭代训练。在公共MRI数据集和自主收集的超声心动图数据上进行详尽实验验证,在多个评估指标方面提供更好的测试结果,但该方法仍耗费较多的标注数据,探索需要更少标注数据的检测算法是将来的工作目标。