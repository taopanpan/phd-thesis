\chapter{医学图像的分割方法}
\label{chap:Segmentation}

 提出了一种基于深度卷积神经网络自动识别超声心动图标准切面的方法,并可视化分析了深度模型的有效性。该算法针对网络全连接层占有模型大部分参数的缺点,引入空间金字塔均值池化替代全连接层,获得更多的空间结构信息,并大大减少模型参数、降低过拟合风险,通过类别显著性区域将类似注意力机制引入模型可视化过程。通过超声心动图标准切面的识别问题案例,试着对深度卷积神经网络模型的鲁棒性和有效性进行了解释。在超声心动图上的可视化分析实验表明,通过改进方法的深度模型的识别决策依据,同医师辨别分类超声心动图标准切面的依据一致,表明了方法的有效性和实用性。

\section{引言}

 
\section{初始位置定位和特征点标注}

 
\section{AAM模型和CLM模型}

\section{结合卷积网络特征的形状对齐模型} 
\subsection{超声组织特征纹理特异性灰度归一化}
\subsection{结合不同外观特征的全局AAM}
\section{实验结果分析和讨论}
\section{本章小结}
本文利用深度学习来解决医学图像计算机辅助检测问题,设计并验证了自动检测MRI短轴和超声心动图中LV长轴切面的方法,在通用物体检测Faster RCNN框架的基础上,针对RPN引入空间变换,结合带朝向损失的多任务损失,探索解决图像平面内物体旋转角度检测的问题,并利用困难样例挖掘策略加快迭代训练。在公共MRI数据集和自主收集的超声心动图数据上进行详尽实验验证,在多个评估指标方面提供更好的测试结果,但该方法仍耗费较多的标注数据,探索需要更少标注数据的检测算法是将来的工作目标。