
\chapter{绪论}
\label{chap:introduction}

\section{研究背景及现实意义}
\subsection{医学图像内容理解的研究背景}

医学影像技术已经彻底变革了医疗卫生系统,成熟的成像模式不断完善,新技术不断涌现,能更早,更有效地诊断身心健康状况,为治疗计划提供信息。用于医疗诊断的成像使医生能够更早地发现疾病并改善患者预后,介入或术中成像有助于消除和治愈许多检测到的疾病。但世界医疗卫生系统每天都会浪费大量的资源和时间,对医学图像内容的错误理解会造成错误诊断,导致很多不必要的额外检查,导致治疗计划的延迟,大大减少了如果早期正确发现的生存或缓解率。

机器学习已被用于医学成像,并将在未来产生更大的影响。从事医学成像的人员必须了解机器学习如何工作。机器学习为医疗行业提供了许多机会,一些医疗保健和技术创新者正在通过实验人工智能(AI)和机器学习来协作并试图改变我们目前的现实。计算机及其运行的算法可以比人类科学家或医学专业人员更快,更准确地提取大量数据,挖掘模式和预测,加强疾病诊断,提供治疗计划,加强公共健康和安全。

机器现在正在学习如何读取CT扫描和其他影像诊断测试来识别异常。虽然有人预测放射科医生的结局,但是也有人认为AI是放射科医生的助手。图像中解剖结构的准确分类和定位是全自动的基于图像的胎盘异常诊断的前兆。对于通常在临床筛查和风险评估诊所获得的胎盘超声图像,这些结构可能具有相当模糊的界限和低对比度,并且即使对于有经验的临床医生来说,图像级解释也是具有挑战性和耗时的任务。机器学习是识别可应用于医学图像的图案的技术。虽然这是一个强大的工具,可以帮助提供医疗诊断,但可能会被误用。机器学习通常始于机器学习算法系统,该系统计算被认为在进行感兴趣的预测或诊断中是重要的图像特征。然后,机器学习算法系统识别这些图像特征的最佳组合,以对图像进行分类或计算给定图像区域的一些度量。有几种方法可以使用,每种方法都有不同的优缺点。大多数这些机器学习方法都有开源版本,使得它们易于尝试应用于图像。测量算法性能的几个指标存在;但是,必须意识到可能导致误导性指标的可能的相关缺陷。最近,深度学习已经开始被使用。这种方法具有不需要图像特征识别和计算作为第一步的优点;相反,功能被识别为学习过程的一部分。

\subsection{课题研究意义}
近来,人工智能(也称为深度学习,机器学习或人工神经网络)将如何帮助临床医生的第一个具体例子现在正在商业化。机器学习软件将作为一个非常有经验的临床助理,提升医生使工作流程更有效率。这些系统可能会为临床医生的工作方式带来范例转变,努力显着提高工作流程效率,同时提高护理和患者的吞吐量。
今天,医生和临床医生面临的最大问题之一就是过多的病人信息过多。电子数据的快速积累归功于电子医疗记录(EMR)的出现以及捕获了以前没有记录的或者至少不容易被数据挖掘的关于病人的各种数据。这包括成像数据,检查和程序报告,实验室值,病理报告,波形,从植入式电生理设备自动下载的数据,从成像和诊断系统本身传输的数据,以及EMR中输入的信息,入院,出院和转移(ADT),医院信息系统(HIS)和计费软件。在接下来的几年里,使用双向病人门户网站将会有更多的数据爆发,病人可以将他们自己的数据和图像上传到他们的EMR中。这将包括与他们的手机拍摄的伤口部位愈合的图像,以减少需要现场后续办公室访问。它还将包括药物依从性跟踪,血压和体重日志,血糖,抗凝剂INR和其他家庭监测测试结果,以及来自应用程序,可穿戴设备和不断发展的物联网(IoT)的活动跟踪,以帮助保持患者健康。
医生们把所有这些数据都比作饮用水,因为它是压倒性的。许多人认为通过大量数据来挑选临床相关或可操作性是非常困难或不可能的。事情很容易通过裂缝掉下来,或者由于病人随访而丢失。如果增加诸如增加患者数量,降低报销额,捆绑支付以及从服务费转换为按价值计费的报销系统等因素,这个问题就更加复杂化了。
这是人工智能在未来几年将发挥关键作用的地方。人工智能不会诊断患者,也不会取代医生 - 这将增加他们找到需要照顾患者的关键数据的能力,并以简洁易懂的格式呈现患者。当放射科医师调用胸部计算机断层扫描(CT)扫描来阅读时,AI将检查图像并立即从图像中识别潜在的发现,并且还通过梳理与所扫描的特定解剖结构相关的患者历史。如果考试顺序是胸痛,那么AI系统会调用:
所有相关的数据和事先检查特定于以前的心脏病史;
有关COPD,心力衰竭,冠心病和抗凝血药物的药物信息;
先前的胸部影像检查可以帮助诊断;
此前报告的成像;
事先胸外或心脏手术;
最近的实验结果;和
任何与从胸部采集的标本相关的病理报告。
以前报道的患者病史或可能与胸痛潜在原因相关的EMR也将由AI收集,并简要显示与全部信息(例如主动脉瘤,高血压,冠状动脉阻塞史,吸烟史,既往肺栓塞,癌症,植入装置或深静脉血栓形成)。否则这些信息将会花费太长的时间来收集,或者医生不能知道其存在,所以他们不会花时间去寻找它。
观看视频“医学影像诊断中的人工智能示例”。这展示了AI如何评估主动脉夹层CT图像的一个例子。
 
观看视频“人工智能的发展援助放射学”,马萨诸塞州综合医院临床数据科学中心主任马克·米歇尔斯基博士的采访,解释人工智能的基础放射学。
在二月的2017年健康信息和管理系统协会(HIMSS)年度会议上,几家供应商展示了这种类型的AI如何工作的一些具体例子。 IBM / Merge,飞利浦,爱克发和西门子已经开始将AI集成到他们的医疗成像软件系统中。通用电气公司使用人工智能的元素来显示预测分析软件,以便在有人呼叫病人或病人数量增加时,对影像科室产生影响。 Vital展示了一个用于成像设备利用率的类似工作中预测分析软件。包括几家分析公司和创业公司在内的其他公司则展示了使用AI快速筛选大量大数据的软件,或者为适当的使用标准提供即时的临床决策支持,最好的测试或成像来进行诊断甚至提供差异诊断。
飞利浦将AI作为其具有自适应智能的新型Illumeo软件的一个组件,该软件可自动获取相关的放射科先前的检查结果。用户可以在特定的MPI视图中点击解剖结构的区域,AI将查找并打开先前的成像研究以显示相同的解剖结构,切片和方向。对于肿瘤学成像,在图像中点击几次肿瘤,AI将执行自动量化,然后对先验进行相同的测量,呈现肿瘤评估的并排比较。这可以显着减少与肿瘤跟踪评估和加速工作流程相关的时间。
在HIMSS 2017上阅读关于AI的博客“2017年HIMSS提供医学范式转变的两种技术”。
 

当放射科医师开始研究时,所有这些信息都以简明的形式呈现,并极大地增强了这位患者的健康状况。爱克发表示,目标是提高放射科医师对患者的理解,从而改善诊断,治疗以及由此产生的患者结果,而不会增加临床医生的负担。
IBM于2015年以10亿美元的价格收购了Merge Healthcare,部分原因是为了在医疗IT市场站稳脚跟。不过,这次购买还给了沃森数百万的放射学研究和大量现有的医疗记录数据,以帮助培训AI评估患者数据,并更好地阅读影像检查。 IBM Watson现在正通过与其他健康IT供应商达成的第三方协议来授权其软件。合同规定,每个供应商都需要用自己的编程为沃森增加额外的价值,而不仅仅是成为经销商。可能这些新合同中最重要的规定是,供应商也需要共享访问他们所能访问的所有患者数据和成像研究。这使得屈臣氏能够继续磨练其数百万个新病人的临床情报。
机器学习的基础知识
需要访问大量的患者数据和图像来提供AI软件算法教育材料以供学习。通过大量的大数据进行排序是AI如何学习临床医生的重要内容,哪些数据元素与各种疾病状态相关并获得临床理解的重要组成部分。这是一个类似的过程,医学生学习绳索,但使用更多的教育输入比人类可以理解的。机器学习软件的第一步就是要学习医学教科书和护理指南,然后回顾一下临床病例。与人类学生不同的是,AI用来学习数百万的数字。
对于AI未能准确判断疾病状态或发现错误或不相关数据的情况,软件程序员在迭代后返回并细化AI算法迭代,直到AI软件在大多数情况下得到正确的结果。在医学中,变量太多,难以总是对人或机器进行正确的诊断。然而,智慧百分比方面,专家们现在认为人工智能软件阅读医学成像研究往往可以匹配,或在某些情况下,超过人类放射科医生。对于罕见的疾病或表现尤其如此,放射科医生在整个职业生涯中只能看到少数这类病例。人工智能的好处是可以从档案中回顾数百甚至数千次这些罕见的研究,使他们精通阅读并确定正确的诊断。而且,与人类思维不同的是,它始终在电脑的脑海中保持新鲜。
人工智能算法通过识别模式来读取类似放射科医生的医学图像。人工智能系统使用大量检查进行训练,以确定来自CT,磁共振成像(MRI),超声或核成像扫描的正常解剖结构。然后使用异常情况训练AI系统的眼睛以识别异常,类似于计算机辅助检测软件(CAD)。然而,与CAD只是放射科医生可能想要仔细研究的区域不同,AI软件具有更多的分析认知能力,基于更多的前几代CAD软件的临床数据和阅读体验。出于这个原因,正在帮助开发医学人工智能的专家经常将认知能力称为“有效的CAD”。
   
人工智能和放射科的下一步
麻省总医院放射科计算和信息科学副主席Keith Dreyer博士表示,深度学习计算机已经在驾驶汽车,监测盗窃金融数据,能够翻译语言并识别基于面部识别的人的情绪。 ,波士顿。他是11月在北美放射学会(RSNA)开幕会议上的主要发言人之一,他在会上讨论了人工智能进入医学成像领域。他还负责其机构开发自己的AI系统,以协助Mass General的医生。
Dreyer解释说:“数据科学革命大约在五年前随着IBM Watson和Google Brain的出现而开始。他说,2012年推出的深度学习算法确实推动了人工智能的发展,到2014年,机器正确读取放射学研究的比例开始下降,准确度达到了95%左右。
德雷尔说AI成像软件并不新鲜,因为大多数人已经在Facebook上使用它来使用面部识别算法自动标记朋友的平台身份。他说训练人工智能是一个类似的概念,在这里你可以开始显示一个电脑的照片猫和狗,它可以训练,以确定在使用足够的图像后的差异。
Dreyer说,人工智能需要大量的数据,强大的计算能力,强大的算法,广泛的投资,然后从编程的角度进行大量的翻译和整合,才能被商业化。
他从放射学的角度说,有两种类型的AI。美国食品和药物管理局已经开始批准的第一种类型是定量AI,只需要510(k)的批准。为临床解释开发的AI将需要FDA的上市前批准(PMA),涉及临床试验。
在机器开始进行初级或同行评审之前,德雷尔说人工智能更有可能被用来回顾旧的检查,以帮助医院找到病人可能没有意识到病情的新病人。他说大约有900万美国人有资格接受低剂量的CT扫描来筛查他们的肺癌。他表示,人工智能可以接受培训,通过在卫生系统中记录的所有先前的胸部CT检查来帮助识别可能患有肺癌的患者。这种类型的回顾性筛查也可能适用于其他疾病状态,尤其是如果AI可以将基因组测试结果拉到狭窄的范围内,使患者易患某些疾病。
他说,总的来说,人工智能提供了一个重要的机会来增强和增强放射科的阅读,而不是取代放射科医生。
德雷耶说:“我们专注于麦克风说话,而我们忽略了病人记录中的所有其他数据。 “我们需要将影像作为患者的另一个数据来源。”他表示,人工智能可以帮助自动进行鉴定,并迅速从电子病历中提取相关患者数据,以帮助诊断或了解患者的状况。
众包治疗选择和监测药物反应

人工智能可以影响医疗保健的另一个领域是收集和分享有关疾病治疗的信息。 Tony Blau博士是一位研究人员,他创办了一家启动社交媒体的机构,将人们分享不同的癌症治疗选择。另一组使用Twitter和Facebook进行药物警戒,作为获取药物试验信息的一种方式,可能没有向行业或管理机构报告。人工智能已经被制药业用于药物化合物的初步筛选,并根据其生物学特性确定哪些药物可能对个体更好。

监视健康流行病

人工智能的影响已经有一些有力的指标来帮助监测和预测世界范围内的健康流行病,有一种情况是计算机算法在世界卫生组织报告前九天确定了埃博拉疫情。电脑通过社交媒体网站,新闻报道和政府网站进行筛选,以确定是否有爆发。与任何算法一样,给出的数据越多,获得的学习就越多,因此未来就越好。尽管目前确定疾病暴发的工作还不完善,但其潜力巨大。

医疗领域的人工智能和机器学习将继续得到改善,影响疾病预防和诊断,通过各种临床试验从数据中提取更多的意义,帮助开发基于个人独特DNA的定制药物,并告知治疗选择等等。
人工智能辅助再现性:几年前,西门子医疗集团率先将人工智能(AI)算法引入心脏回波系统,以加速自动化。几年前,飞利浦医疗保健公司也在其Epiq超声系统中引入了AI的元素。它需要一个三维回波数据集采集和自动分析图像,以确定心脏的解剖,标签,然后切片的最佳标准视图呈现。这消除了互操作性差异的问题,因为软件将总是选择基于机器学习的最佳视图,该机器学习使用数千个代表患者解剖变异谱的先前检查。这对于操作人员来说要积累相同的知识需要花费数年的时间。其他供应商也引入了深度学习算法的元素来帮助分析超声心动图或执行自动量化。下一代回声系统将结合更多的人工智能功能,通过自动完成耗时的任务和扩大超声检查员的工作量,从而进一步改善工作流程,从而提高工作效率,始终保持准确。
所有主要的成像系统供应商都在开发他们自己的AI或与AI供应商合作。西门子医疗集团今年早些时候在HIMSS 2017上宣布与IBM Watson建立合作伙伴关系。通用电气医疗集团今年5月宣布将与Partners HeathCare合作,该合作伙伴将通过新成立的麻省总医院和布里格姆妇女医院临床数据科学中心执行。 Mass General一直在开发自己的放射科AI系统。除了Epic回声软件之外,飞利浦还开发了自己的AI,以支持其IntelliSpace Enterprise医学成像信息平台,该平台非常智能,可以将所有患者的相关先前检查结果放在相同的解剖结构中,并以完全相同的视图打开图像作为目前的考试。            
\section{国内外研究现状及难点}
\subsection{医学图像内容理解的研究现状}
通用电气,西门子和飞利浦是超声心动图供应商之一,将深度学习算法整合到回声软件中,帮助自动从三维超声数据集提取标准成像视图。这是飞利浦Epiq系统的一个例子,该系统使用供应商的解剖智能软件来定义解剖结构,并自动显示解剖标准诊断视图,无需人工干预。这可以大大加快工作流程并减少操作员之间的差异。
基于人工智能(AI)的医学图像分析采用卷积神经网络,支持向量机,模糊逻辑系统等机器学习方法从医学图像中提取意义。最先进的计算机视觉软件为诊断人员提供了基于证据的技巧,消除了可能的疑惑并确保了诊断的一致性。
标准视图位置是超声心动图中的关键步骤,因为这些帧包含基本的诊断数据。从超声波检查自动捕捉标准飞机可以加快扫描,并使其更加准确。仔细研究这方面的研究将证明这不是一个猜测。标准视图的计算机辅助检测不断支持临床医生。
计算机如何看到图像
医学图像分析是计算机视觉的实际应用 - 计算机科学的一个分支,涉及数字图像(包括数字视频帧)中的对象和特征识别。计算机视觉算法通过一系列过程来分析图像,类似于人类视觉系统所执行的过程。在经过初步预处理(包括去噪,滤波和特征增强)之后,软件在图像分割的过程中将图像分解成有意义的区域。然后,算法提取重要的特征,并基于这些特征对图像中的对象进行分类。此外,医学图像分析算法通常执行图像配准 - 映射两个以上相同解剖结构的图像以检测任何差异或变化。
基于机器学习,分类是医学图像分析软件最复杂的功能。每个AI系统都使用机器学习方法作为其“大脑”。这些算法允许计算机记住大量信息,并在学习完成后使用它来分析类似的信息。这就是为什么这种方法在计算机视觉中得到如此广泛的应用 - 在图像数据集(例如超声图像数据集)上进行训练,然后软件识别真实世界图像中的熟悉特征(例如,在实时超声扫描中)和在此基础上作出相关的结论。
这些系统的准确性随着输入数据的数量而增加。从数百个图像开始,它们显示出不错的结果,并且在处理了数以千计的图像和更多图像之后,它们的准确度接近100%。当然,这也取决于所使用的架构,随着机器学习方法的发展,用于医学图像分析的算法显示出更好的结果。
计算机视觉在超声心动图中的应用
心脏回声有一些挑战,医学图像分析可以解决。例如,研究人员建议使用计算机视觉自动分割解剖结构,检测和分类先天性心脏缺陷,实时导管定位等。标准视图采集是心脏超声最基本的任务,也可以通过医学图像分析。
\subsection{人工智能在医学图像领域的原理及研究现状}
标准视图获取
为了找到标准的心脏视图,软件应该从超声波扫描期间的多个帧中选择合适的二维平面。在这里,出现了不同的挑战,如分析二维帧,三维体积,二维时间序列或四维时空图像相关(STIC)体积。
国际合作已经解决了后一个问题,提出了使用尺度不变特征变换(SIFT),最先进的特征检测算法和支持向量机(SVM),监督机器学习方法。此方法已在包含正常和异常情况的数据集上进行了测试。软件定位三个标准平面:四腔切面(A4C),三切面切面(TVV)和横切腹切面(TAS)。
该方法在综合数据上显示出了很好的效果,即在随机选择的飞机中检测到了标准的心脏视图,准确率达87-100%。但是,在实际数据上,表现中等,精确度为33-53%。不过,在这两个数据集上,该方法比之前的方法显示出更大的结果。未来,研究人员计划通过应用深度卷积神经网络(CNN)来提高准确性,CNN是用于图像分析的最有前途的机器学习方法。另一个合作提出了一个类似的方法:他们已经应用了基于两个CNN的融合深度学习框架来定位三维回波中的八个标准心脏视图,并且达到了92.1%的准确度。当仅定位三个主要飞机时,准确度高达98%。
值得注意的是,这两项研究都使用相对较少的数据来训练他们的系统:他们向SVM和CNN系统输入了对应于数百个超声平面图像的数据。这可能足以测试一个系统的性能,但经过对大数据集的严格培训后,机器学习软件将显示出更好的结果。
最新技术和期望
今天,CNN被认为是机器学习中最强大的分类技术。专门设计来分析图像,他们显示壮观的图像分类准确性。在一些任务中,他们已经超越了人类,正如年度ImageNet视觉识别挑战所显示的那样。获奖的ImageNet研究团队拥有数百万个标记图像来训练他们的卷积神经网络。因此,随着医学图像数据量的不断增长,我们可以期待医学图像分析软件很快成为超声系统的重要组成部分。
\section{全文结构及创新点}

\section{本章小结}