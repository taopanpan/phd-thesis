\chapter{空间金字塔分解的深度可视化方法}
\label{chap:visualization}

 提出了一种基于深度卷积神经网络自动识别超声心动图标准切面的方法,并可视化分析了深度模型的有效性。该算法针对网络全连接层占有模型大部分参数的缺点,引入空间金字塔均值池化替代全连接层,获得更多的空间结构信息,并大大减少模型参数、降低过拟合风险,通过类别显著性区域将类似注意力机制引入模型可视化过程。通过超声心动图标准切面的识别问题案例,试着对深度卷积神经网络模型的鲁棒性和有效性进行了解释。在超声心动图上的可视化分析实验表明,通过改进方法的深度模型的识别决策依据,同医师辨别分类超声心动图标准切面的依据一致,表明了方法的有效性和实用性。

\section{引言}

 
\section{梯度更新的可视化方法}

 
\section{空间金字塔分解}
\subsection{高斯和拉普拉斯金字塔分解}
\subsection{梯度归一化}
 
\section{实验结果分析和讨论}
\section{本章小结}
本文针对理解深度CNN特征空间存在的问题,
提出一种用于改善深度 CNN 分类模型的可视化方 法. 其中通过改善激活最大化可视化技术来产生更 具有全局结构的细节、上下文信息和更自然的颜色 分布的高质量图像. 该方法首先对反向传播的梯度 进行归一化操作,在常用正则化技术的基础上,提出 使用空间金字塔分解图像不同频谱信息;为限制可 视化区域,提出利用类别显著激活图技术,可以减少 优化产生重复对象碎片的倾向,而倾向于产生单个 中心对象以改进可视化效果. 激活最大化可显示
CNN在分类时关注什么. 这种改进的深度可视化技 术将增加我们对深层神经网络的理解,进一步提高 创造更强大的深度学习算法的能力. 该方法适用于 基于梯度更新的可视化领域,是对网络模型整体的 理解,具体各层特征怎么耦合成语义信息仍需进一 步探索,深度CNN模型如何重建一个完整的类别概念,仍是一个开放性问题.